\documentclass[../main/Self-Stabilization.tex]{subfiles}
\begin{document}
\section{Introduction}
The Internet of Things (IoT) has been a highly discussed topic in he past 10 years within many areas of research, however it dates back in 1990's, precisely with the start of Industrial automation systems \cite{prasad2012energy}. IoT is a term used to describe a network of ``things''; smart objects (devices), not traditionally thought of as computers (e.g., cars, household appliances). They are equipped with some sort of sensors, and are normally connected to the world surrounding them (e.g. human beings, other objects or sensors), being able to exchange information with other objects using Internet protocols and technologies (TCP/IP, etc.).

The most interesting aspect of the IoT is it's future impact on our daily life. Zorzi et al \cite{zorzi2010today} points out there is a first time opportunity to interact with the surrounding environments and to exchange information that previously was not available by simply looking at objects (devices). Apart from a direct interaction individual-machine (object), it will also have a indirect effect through object-to-object interaction. Prasad and Kumar \cite{prasad2012energy} describes this particular aspect of IoT as the advance version of Machine to Machine (M2M) communication, where objects are exchanging information between them without a human intervention.

The application of IoT is vast, covering many areas of research including Smart Cities (with Traffic Management research, Smart Parking etc.), Smart Environment (with Air Pollution research, Fire Detection research etc.) and so on. Including billions of objects interconnected that can communicate with each other and act according to their environment, the Internet of Things will be useful in particular in resource-constrained systems (e.g. with smart grids).

The naive approach to an Internet of Things would use a central controller or master node of some sort to oversee the activities of all ``things'' in the Internet of Things.  However, this has obvious drawbacks, not the least being scalability. It is therefore desirable that a ``thing'' govern its own actions while achieving global ``self'' properties (such as self-adaptivity, self-stabilization) in an Internet of Things that has to work with resource constraints (e.g., limited allowable peak electricity usage for a domestic or industrial system of many ``things''). Such an Internet of Things with self-governing ``things'' would not require a central controller, and addition or removal of ``things'' would be far easier.


\subsection{Resource-constrained systems and Solar energy}
text
\subsection{Limitations of using a central controller of the system}
more text
\subsection{Self-adaptivity. Self-stabilization}
more text
\end{document} 