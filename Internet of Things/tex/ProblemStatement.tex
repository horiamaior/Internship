\documentclass[../main/IoT.tex]{subfiles}
\begin{document}

\section{Problem Statement and The Model}\label{problem}
In Section \ref{intro} we discussed and presented our interest and in the same time the focus of this paper. In this section we will describe an abstract model of IoT within resource-constrained domain, formally stating the problem we address. We will develop algorithms working on the model developed and we will attempt to prove the correctness of the main properties of our system.

\subsection{Definition and Notations}
Let there be numerous objects $o_{i}$ that demand and may consume power resource. The set of all objects, denoted as $O$, together with some sort of power supply constitute the whole system.
 \begin{center}
    $O=\{o_{0}, o_{1}, ...,o_{n-1}\}$, the set of all $n$ objects.
 \end{center}
Each object consumes a non-negative amount of power resource. To simplify the system, let us consider this non-negative amount of power constant for each object in the system (an object cannot change its power demand). Let $\mathbb{R^+}$ be the set of positive real numbers, then let the demand function, $f:O\rightarrow\mathbb{R^+} | f(o_{i})=r$, give the power demand of each object $o_{i}$. This means object $o_{i}$ demands $r$ amount of power from the power supply.

The Total Demand of system at any one time, as described by Rao~\cite{rao2011foundation}, is the sum of all power demands for all objects in the system:
\begin{center}
  %$\sum_{i=0}^{n-1} o_i$
   $\sum\limits_{i=0}^{n-1} f(o_i)$
\end{center}

The power supply can be described as the total amount of power resource available to share between all the objects of the system at any time. The main property of the power supply is that it is variable in time. Let the set $T$ denote the set of all time instances, and $\mathbb{R^+}\cup\{0\}$ be the set of positive real numbers including $0$. Then, the function $\gamma:T\rightarrow\mathbb{R^+}\cup\{0\}$ denotes the power supply function. With $\gamma(t)= w$, $w$ the \cite{rao2011foundation} the maximum resource limit, in our case power resource, available in the system at a given time $t$.

We described objects having various power demands and we described a power supply providing the system with power resource. As presented in the introduction, objects also differ between them in terms of some sort of priority.

Let $\delta:O\times T\rightarrow\mathbb{R^+}\cup\{0\}$ be the priority function, with $\delta(o_{i}, t)= p$, where $p$ is the priority of object $o_{i}$ at time $t$. As described, priority function is time dependent, that means objects can change priority in time.

\subsection{IoT as a Distributed System}

In a distributed IoT, every object is networked in some way so it is able to exchange information with all other objects in the system (for simplicity, we make abstraction of how they are physically networked). It is also assumed that every object has some sort of computing capability (we assume that computing capability has no cost what so ever), and it is in some way `self-aware' of its current `needs' (for example each object knows or is able to find out about its power settings). Because we are building a de-centralized system, the decision-making comes to the object itself (this means that each object in the system decides on its own when is a suitable time to consume power resource), therefore, every object is also interested in an overview of the whole system (for example each object needs to know the maximum power limit $\gamma(t)= w$ but also each object has to know about what is its priority relative to other's in the network).

When creating such a system, each object in the system is ``exploring'' the whole system in some way. During the exploration, objects need to find out (and memorize in some way) other objects with the same priority, and in the same time find out other priorities in the system. This way, every object will have an general overview of what is their own priority relative to other object in the system. The exploration is possible trough some sort of communication and exchange of information between objects. Making abstraction of how the communication is made, let $v$ and $u$ be two objects; $v,u \in\{O\}$. Let $msg(v,u,m)$ be a message, where $v$ is the sender of the message, $u$ is the delivery object of the message and $m$ is the message (the message can contain any kind of information coming from the sender). Following Peleg's approach \cite{peleg2000distributed}, a message can be also broadcasted or ``flooded'' to/over a network (in our case system). Algorithm \ref{flood} below, is an adapted version from \cite{peleg2000distributed}, for broadcasting a message across all $n$ objects of a system, from a root object $o_{j}$:

%Algorithm 1 - Flooding
\LinesNumbered
\IncMargin{1em}
\begin{algorithm}
%\DontPrintSemicolon
Let a source $o_{j}$
\BlankLine
\For{$i \gets 0$ \textbf{to} $n-1$} {
    \If {$i!=j$} {	
        \textbf{Send} \emph{$msg(o_{j},o_{i},m)$}
    }
}
\For{$i \gets 0$ \textbf{to} $n-1$} {
    \textbf{Upon} receiving a message $o_{i}$ \newline
       \textbf{Store} the message\newline
       \textbf{Compute} the message\newline
       \textbf{Send} acknowledgement
}
\caption{\textbf{Algorithm Flood}} \label{flood}
\end{algorithm}
\DecMargin{1em}

The number of messages sent by each object can be used to evaluate the complexity of the Algorithm \ref{flood} and all further algorithms developed in this paper. Each object running Algorithm \ref{flood} is sending a message to all other objects (in total $n-1$ objects); therefore the complexity of Algorithm \ref{flood} is $\mathcal{O}(n)$.

Using Algorithm \ref{flood}, we design an exploration algorithm that is run by all objects in the system, all trying to communicate to other objects information about their priority and their power demand. In the same time, objects receive, compute and store information about other objects. Therefore, every object will end up having an overview of their position in the system.

Let $o_{i}$ be an object; $o_{i} \in\{O\}$. As described above, each object stores information about other priorities in the system and other objects of the same priority in the system. Let each every object $o_{i}$ have to arranged lists:
\begin{itemize}
    \item let $P_{i}=$ be an arranged list where each object $o_{i}$ can store priorities of other objects (decreasing order); such that $P_{i}(0)$ is the highest priority in the set.
    \item let $Q_{i}$ be an arranged list of objects of the same priority as $o_{i}$ (increasing order); such that $Q_{i}(0)$ is the object with smallest demand.
\end{itemize}

Before joining the system and running the exploration algorithm, every object $o_{i}$ has: $P_{i}=\{\emptyset\}$ and $Q_{i}=\{\emptyset\}\cup\{\delta(o_{i},t)\}$. Please consider Exploration Algorithm (\ref{algo2})below.

%Algorithm 2 - exploration
\LinesNumbered
\IncMargin{1em}
\begin{algorithm}
%\DontPrintSemicolon
$t_{0} = current time$
\BlankLine
$o_{i} \gets$ Flood Algorithm over $G$ \newline
\textbf{Send}: $msg(o_{i}, join, \delta(o_{i}, t_{0}), f(o_{i}))$
\BlankLine
\For{$j \gets 0$ \textbf{to} $n-1$} {
    Let $o_{j}$, ($j!=i$) receive the message
    \BlankLine
    \textbf{Upon} receiving message:\newline
    \If {$\delta(o_{i},t_{0}) = \delta(o_{j},t_{0})$}
    {
       \textbf{Insert} $o_{i}$ in $Q_{j}$ \textbf{in order} of $f(o_{i})$
       \BlankLine
       \textbf{Send} acknowledgement ($\delta(o_{i},t_{0}) = \delta(o_{j},t_{0})$)
    }
    \Else {
        \If {$\delta(o_{i},t_{0}) \notin P_{j}$} {
            \textbf{Insert} $\delta(o_{i},t_{0})$ in $P_{j}$ \textbf{in order}
            \BlankLine
            \textbf{Send }acknowledgement $\delta(o_{i},t_{0}) < \delta(o_{j},t_{0})$ (if so) \newline
            \textbf{Send }acknowledgement $\delta(o_{i},t_{0}) > \delta(o_{j},t_{0})$ (if so)
        }
    }
}
\caption{\textbf{Exploration Algorithm} run by all object joining the system} \label{algo2}
\end{algorithm}
\DecMargin{1em}

The Algorithm \ref{algo2} has complexity $\mathcal{O}(n^2)$, and this is mostly because there are $n$ objects running the Algorithm \ref{flood}, every object sending $n-1$ messages (that is $n\times n$).

Being an extensible IoT system, new objects can join at any time, and objects may no longer want to be part of the system and leave at any time. An object joins the system when it demands power resource. When joining, the new object shall run the exploration algorithm. This way, other objects will make note of the new object's details (like priority and power setting), but in the same time, the new object will create it's overview of the system trough message acknowledgements from the existing objects. An object is leaving the system when it is no-longer interested in power resource at that particular time. On leaving, the object informs all the other objects about it's intentions.

Let $B = \{T,F\}$ the set of boolean values true ($T$) and false ($F$), and function $p:O\times T \rightarrow B$:

  \[ p(o_{i}, t)= \left\{
  \begin{array}{l l}
    $T$ & \quad \textrm{if $o_{i}$ is powered}\\
    $F$ & \quad \textrm{if $o_{i}$ is not powered}
  \end{array} \right.\]
When an object $o_{i}$ gets powered at time $t$, the boolean function $p(o_{i},t) \gets T$, becomes true. Algorithm \ref{algo3} below describes exactly what happens when an object is leaving the system.


%Algorithm 3 - removing an object
\LinesNumbered
\IncMargin{1em}
\begin{algorithm}
%\DontPrintSemicolon
$t = leaving time$
\BlankLine
$o_{i} \gets$ leave the system \newline
Flood Algorithm over $G$ \newline
\textbf{Send} $msg(o_{i}, leave,\delta(o_{i}, t), Q_{i}, f(o_{i}), powered=p(o_{i},t))$
\BlankLine
\For{$j \gets 0$ \textbf{to} $n-1$} {
    Let $o_{j}$, ($j!=i$) receive the message
    \BlankLine
    \textbf{Upon} receiving:\newline
    \If {$\delta(o_{i},t) = \delta(o_{j},t)$} {
          \textbf{Remove} $o_{i}$ from $Q_{j}$
    }
    \Else {
        \If {$Q_{i} - \{o_{i}\} = \emptyset$}{
            \textbf{Remove} $\delta(o_{i},t)$ from $P_{j}$
        }
    }
    \If{$powered = T$}{
        $w \gets w + f(o_{i})$
    }
}
\caption{\textbf{Leave Algorithm} run by every object leaving the system} \label{algo3}
\end{algorithm}
\DecMargin{1em}
On leaving, an object sends a message to all other objects in the system. Like Algorithm \ref{flood}, Leave Algorithm has the complexity $\mathcal{O}(n)$.

In Algorithm \ref{algo3}, the leaving object is messaging all other objects three important pieces of information:
 \begin{itemize}
   \item it's priority level $\delta(o_{i}, t)$ over (together with the list with other objects of it's priority $Q_{i}$)
   \item it's power demand $f(o_{i})$.
   \item whether or not the leaving object was powered (function $p(o_{i}$)).
 \end{itemize}

In the case that the leaving object has the same priority with the one receiving the message, the last one mentioned has to remove the the leaving object from the same priority objects database $Q_{j}$ (see line $5, 6$ in Algorithm \ref{algo3}). In the case that the leaving object is the last of its priority, the whole priority level is removed from the other priorities database $P_{j}$ (see line $9, 10$ in Algorithm \ref{algo3}). Finally, if the leaving object was powered, it's power demand becomes available for other objects (see line $13-15$ in Algorithm \ref{algo3}).

\subsection{Prioritized objects and a happy IoT}

In he subsections above, it has been described a distributed, extensible system, with objects that are able to communicate with other objects of the system, with new objects joining the system and objects maybe leaving the system. However, we did not yet mentioned how the system is powered from the power supply, that is how each individual object receives, if so, power resource.

We discussed that objects can have different priorities, and that is because some objects are more important in a way than others. In the case of not being able to satisfy all objects with power resource, higher priority objects are more likely to get power resource than lower-priority ones. We also discussed that objects can have different power demands. That is, one object might need more power resource than other ones. In the case of objects with the same priority level, the system shall always prioritize the low consumer object.

Let there be $n$ objects $o_{i}\in O$, with $0<=i<=n-1$. Each object has a power demand $f(o_{i})$, a priority function $\delta(o_{i}, t)$, and each object has created its own overview of other objects priorities in the system (in decreasing order the list $P_{i}$; such that $P_{i}(0)$ the highest priority), and objects with the same priority (in increasing order of power demands the list $Q_{i}$; such that $Q_{i}(0)$ is the smallest demand object of $o_{i}'s$ priority). The power supplies is denoted by the function $\gamma(t) = w$, providing the power resource available at every time $t \in T$. The following algorithm (Algorithm \ref{algo4}) describes how the power supplier, supplies the highest priority object with the smallest power demand first, and then how each object informs other objects about their actions and remaining power resource.

%Algorithm 4 - powering the system
\LinesNumbered
\IncMargin{1em}
\begin{algorithm}
%\DontPrintSemicolon
$t = current time$
\BlankLine
Let $o_{i}$ the object running this code
\BlankLine
\If {$\delta(o_{i}, t)>P_{i}(0)$ \textbf{and} $f(o_{i})=Q_{i}(0)$}{
    $w \gets \gamma(t)$
    \BlankLine
    \If{$f(o_{i})<=w$}{
        \textbf{Power} object $o_{i}$;
        $p(o_{i},t) \gets T$
        \BlankLine
        $w \gets w - f(o_{i})$
        \BlankLine
    }
    \If {$Q_{i} - \{o_{i}\} = \{\emptyset\}$}{
        $nextPriority \gets P_{i}(0)$
        \BlankLine
        \textbf{Broadcast} $msg(power, nextPriority, w)$
    }\Else{
        $nextObj \gets Q_{i}(1)$
        \BlankLine
        \textbf{Send} $msg(power, nextObj, w$)
    }
}
\textbf{Upon} receiving one of the msg:\newline
\textbf{Receiving} $msg(nextObj, w)$ or\newline
\textbf{Receiving} $msg(nextPriority, w)$\newline
\If{$(o_{i} = nextObj$)\textbf{ or }\newline
    $(\delta(o_{i}, t) = nextPriority$ \textbf{and} $f(o_{i})=Q_{i}(0))$}{
    \If{$f(o_{i})<=w$}{
        \textbf{Power} object $o_{i}$;
        $p(o_{i},t) \gets T$
        \BlankLine
        $w \gets w - f(o_{i})$\newline
    }
    \If {$Q_{i} - \{o_{i}\} = \{\emptyset$\}}{
        $nextPriority \gets P_{i}(j) <=>$ \newline
        $(P_{i}(j-1)>\delta(o_{i},t) > P_{i}(j)$)
        \BlankLine
        \textbf{Broadcast} $msg(nextPriority, w)$
    }\Else{
        $nextObj \gets Q_{i}(j) <=> (o_{i}=Q_{i}(j-1))$
        \BlankLine
        \textbf{Send} $msg(nextObj, w)$
    }
}
\caption{\textbf{Power Algorithm}} \label{algo4}
\end{algorithm}
\DecMargin{1em}

Algorithm \ref{algo4} is run by all objects in the system individually. The first part of the algorithm, lines $3-17$, will addresses to the object with the highest priority and the lowest power demand of the system (line $3$), this object having the first chance to get powered (in the case that the available power resource can satisfy its power demands - line $4-6$). From this point, the current object informs the next object with the same priority (if there is one - lines $13-16$), otherwise it broadcasts an information containing the new priority that shall receive power (lines $9-12$). The second part of Algorithm \ref{algo4}, lines $18-31$ address all the other objects of the system, which are waiting for a message addressing them or their priority level. The smallest consumer of a priority category will always be powered first. Regardless if an object gets powered, it will always send a message to the next object or priority group. This enables small demand objects with small priority to get powered in the case of a high priority object cannot fulfil its demands with available power resource.

The complexity of Algorithm \ref{algo4} is $\mathcal{O}(n^2)$. The worse case scenario is when all objects have different priorities, and every object is broadcasting a message once ($n\times n$).

\subsection{Proofs of Correctness}
In the previous subsections we developed a framework and created algorithms that describe the behaviour of objects within the IoT model. In this subsection, we are identifying a set of properties of the model, and we are trying to prove the correctness of our proposed design by validating these properties.
\newtheorem{theorem}{Theorem}[section]
\newtheorem{lemma}[theorem]{Lemma}
\begin{theorem}
\emph{No Wastage}. Guarantying that all the available power will be used by the objects of the system, and there will be not object entering a situation where there is available resource in the system meeting that objects demands but the object is not powered.
\end{theorem}

\begin{lemma}\label{lemma1}
  After running Algorithm \ref{algo4}, $\nexists o_{i}\in O, (p(o_{i},t) = F)$ \textbf{and} $(f(o_{i})\leq w); $
\end{lemma}

\begin{proof} Let us assume $\exists o_{i} \in O$ such that $p(i,t)=F$ while there is sufficient power to satisfy its demands, $(f(o_{i})\leq w)$. There are two cases to consider:

\begin{itemize}
  \item object $o_{i}$ is the highest priority object with the smallest power demand (between objects of the same priority); if $p(o_{i},t)=F$, following line $5,6$ in Algorithm \ref{algo4}, $(f(o_{i})\nleq w)$. Power demand $(f(o_{i})$ cannot be both $\leq\nleq w$. Hence the contradiction.
  \item Otherwise, object must have received a message from a higher priority object; in this case the proof is similar; if $p(o_{i},t)=F$, following line $19, 20$ in the same Algorithm \ref{algo4}, $(f(o_{i})\nleq w)$. Power demand $(f(o_{i})$ cannot be both $\leq\nleq w$.
\end{itemize}
\end{proof}


\begin{theorem}
\emph{No Priority Inversion}. Guarantying that high priority objects get powered first, if possible.
\end{theorem}
\begin{lemma}\label{lemma2}
      $\nexists u,v \in O$ such that ($\delta(u,t) \le \delta(v,t)$) \textbf{and} ($w\geq f(u)$ and $w\geq f(v)$), ($ p(u,t)=T$ \textbf{and} $p(v,t)=F$).
\end{lemma}

\begin{proof}
    If we assume $\exists u, v \in O$, such that priority of object $\delta(u,t) \le \delta(v,t)$ and the power budget can satisfy any of the two objects with power resource $w\geq f(u)$ and $w\geq f(v)$. We also assume that $p(u) = T$ \textbf{and} $p(v,t) = F$ after running Algorithm \ref{algo4} by both objects. There are two cases here:
    \begin{itemize}
      \item object $v$ is has the heights priority with the smallest power demand; following line $3-6$ in Algorithm \ref{algo4}, object $v$ gets powered first and $p(v,t) = T$. Object $v$ cannot be in the same time both powered $p(v,t) = T$ and not powered (the assumption of the proof) $p(v,t) = F$
      \item neither of the objects $u,v$ are the have the highest priority with the smallest power demand; objects get powered when receiving a message dedicated for them. Message flow is from the highest priority object (line $3$) towards the lowest priority object (line $9-16$). If , $p(u) = T$ and $p(v) = F$, the message must have first visited object $u$. Therefore $\delta(u,t) \ge \delta(v,t)$. But the priority comparison cannot be both $\delta(u,t) \le \delta(v,t)$ and $\delta(u,t) \geq \delta(v,t)$ in the same time. Hence the contradiction.
    \end{itemize}
    because the message flow (with the remaining power resource) starts from the highest priority object (line $3$) towards the lowest priority object (line $9-11$). Therefore the message with the remaining power resource must have first visited object $v$, therefore the contradiction.
\end{proof}
\end{document} 