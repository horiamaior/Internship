\documentclass[../main/IoT.tex]{subfiles}
\begin{document}
\begin{abstract}
  The Internet of Things is a term to describe networked objects not traditionally thought of as computers (e.g., cars, household appliances) which may nonetheless be connected using Internet protocols and technologies (TCP/IP, etc.). Such ``things'' may also be connected, for control or communication, to the traditional Internet, and may themselves also be equipped with sensors or actuators to interact with their environments. It is envisioned that an Internet of Things will be useful in particular in resource-constrained systems (e.g., with smart grids). The naive approach to an Internet of Things would use a central controller or master node of some sort to oversee the activities of all ``things'' in the Internet of Things.  However, this has obvious drawbacks, not the least being scalability. It is therefore desirable that a ``thing'' govern its own actions while achieving global ``self'' properties (such as self-adaptivity, self-stabilization) in an Internet of Things that has to work with resource constraints (e.g., limited allowable peak electricity usage for a domestic or industrial system of many ``things''). Such an Internet of Things with self-governing ``things'' would not require a central controller, and addition or removal of ``things'' would be far easier. This paper provides a theoretical model of such Self-Internet of Things, giving a set of principles and properties to the system in respect to resource-constrain and energy efficiency systems.  This would involve a proposed type of behaviour for a single ``thing'' in such an Internet of Things, as well as the principles or protocols by which such ``things'' are connected with one another.\end{abstract}
\end{document} 